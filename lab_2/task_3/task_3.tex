% Preamble
\documentclass[11pt]{article}

% Packages
\usepackage[T1]{fontenc}
\usepackage[utf8]{inputenc}
\usepackage[polish]{babel}
\usepackage{amsmath}
\usepackage{algorithm2e}
\usepackage{algpseudocode}
\usepackage[a4paper]{geometry}

% Translations
\SetKw{KwTo}{do}
\SetKw{Length}{długość}
\SetKw{ToNumber}{liczba}
\SetKwFor{For}{Dla}{wykonuj}{koniec}
\SetKwFor{While}{Gdy}{wykonuj}{koniec}
\SetKw{Continue}{Przejdź do kolejnej iteracji pętli}
\SetKwIF{If}{ElseIf}{Else}{Jeżeli}{to}{w przeciwnym razie jeżeli}{w przeciwnym razie}{koniec}

% Document
\begin{document}

    \section{Opóźniony generator pseudolosowy Fibonacciego.}

    \begin{algorithm}[H]
        \SetAlgoLined
        Start\\
        Wczytaj \textbf{n}, współczynniki \textbf{j}, \textbf{k}, operację \textbf{op} i wartości początkowe \textbf{x}\\
        l $\gets$ długość(x)\\
        \For{i $\gets$ 1 \KwTo n}{
            x1 $\gets$ x[(i-j+l) mod l]\\
            x2 $\gets$ x[(i-k+l) mod l]\\

            result $\gets$ op(x1,x2)\\

            x[i mod l] $\gets$ result\\
            Wypisz \textbf{result}\\
        }
        Stop\\

        \label{alg:algorithm}
    \end{algorithm}

\end{document}