% Preamble
\documentclass[11pt]{article}

% Packages
\usepackage[T1]{fontenc}
\usepackage[utf8]{inputenc}
\usepackage[polish]{babel}
\usepackage{amsmath}
\usepackage{algorithm2e}
\usepackage{algpseudocode}
\usepackage[margin=1in]{geometry}

% Translations
\SetKw{KwTo}{do}
\SetKw{Length}{długość}
\SetKw{ToNumber}{liczba}
\SetKwFor{For}{Dla}{wykonuj}{koniec}
\SetKwFor{While}{Gdy}{wykonuj}{koniec}
\SetKw{Break}{Wyjdź z pętli}
\SetKw{Continue}{Przejdź do kolejnej iteracji pętli}
\SetKwIF{If}{ElseIf}{Else}{Jeżeli}{to}{w przeciwnym razie jeżeli}{w przeciwnym razie}{koniec}

% Document
\begin{document}

    \section{Wyszukiwanie wzorca. Algorytm Karpa-Rabina.}

    \begin{algorithm}[H]
        \SetAlgoLined
        Start\\

        Wczytaj $text$, $pattern$ oraz funkcję hashująca $hf$\\

        $text\_length \gets$ \Length $text$\\
        $pattern\_length \gets$ \Length $pattern$\\

        $matches \gets$ pusta tablica\\
        $pattern\_hash \gets hf(pattern)$\\

        $i\_max \gets text\_length - pattern\_length + 1$\\

        \For{$i \gets 0$ \KwTo $i\_max$}{
            $slice\_text \gets ""$\\

            \For{$j \gets 0$ \KwTo $pattern\_length$}{
                $slice\_text \gets slice\_text + text[j]$\\
            }

            $slice\_hash \gets hf(slice\_text)$\\

            \If{$pattern\_hash = slice\_hash$}{
                Wstaw wartość $i$ do listy $matches$\\
            }
        }

        Wypisz listę $matches$\\
        Stop\\
        \label{alg:algorithm1}
    \end{algorithm}

\end{document}